%%% Exemplo de utilização da classe ITA
%%%
%%%   por        Fábio Fagundes Silveira   -  ffs [at] ita [dot] br
%%%              Benedito C. O. Maciel     -  bcmaciel [at] ita [dot] br
%%%              Giovani Volnei Meinertz   -  giovani [at] ita [dot] br
%%%    	         Hudson Alberto Bode       -  bode [at] ita [dot]br
%%%    	         P. I. Braga de Queiroz    -  pi [at] ita [dot] br
%%%    	         Jorge A. B. Gripp         -  gripp [at] ita [dot] br
%%%    	         Juliano Monte-Mor         -  jamontemor [at] yahoo [dot] com [dot] br
%%%    	         Tarcisio A. B. Gripp      -  tarcisio.gripp [at] gmail [dot] com
%%%
%%%
%%%  IMPORTANTE: O texto contido neste exemplo nao significa absolutamente nada.  :-)
%%%              O intuito aqui eh demonstrar os comandos criados na classe e suas
%%%              respectivas utilizacoes.
%%%
%%%  Tese.tex  2016-08-25
%%%  $HeadURL: http://www.apgita.org.br/apgita/teses-e-latex.php $
%%%
%%% ITALUS
%%% Instituto Tecnológico de Aeronáutica --- ITA, Sao Jose dos Campos, Brasil
%%%                   http://groups.yahoo.com/group/italus/
%%% Discussion list: italus {at} yahoogroups.com
%%%
%++++++++++++++++++++++++++++++++++++++++++++++++++++++++++++++++++++++++++++++
% Para alterar o TIPO DE DOCUMENTO, preencher a linha abaixo \documentclass[?]{?}
%   \documentclass[tg]{ita}			= Trabalho de Graduacao
%   \documentclass[tgfem]{ita}	= Para Engenheiras
%   								msc     		= Dissertacao de Mestrado
%   								mscfem   		= Para Mestras
%   								dsc      		= Tese de Doutorado
%   								dscfem   		= Para Doutoras
%   								quali    		= Exame de Qualificacao
%   								qualifem 		= Exame de Qualificacao para Doutoras
% Para 'Draft Version'/'Versao Preliminar' com data no rodape, adicionar 'dv':
%   \documentclass[dsc, dv]{ita}
% Para trabalhos em Inglês, adicionar 'eng':
%   \documentclass[dsc, eng]{ita}
%		\documentclass[dsc, eng, dv]{ita}
%++++++++++++++++++++++++++++++++++++++++++++++++++++++++++++++++++++++++++++++
\documentclass[mscprof]{ita}    % ITA.cls based on standard book.cls
    % Quando alterar a classe, por exemplo de [msc] para [msc, eng]) rode mais uma vez o botão BUILD OUTPUT caso haja erro
    \usepackage{ae}
    \usepackage{graphicx}
    \usepackage{epsfig}
    \usepackage{amsmath}
    \usepackage{amssymb}
    \usepackage{subfig}
    \usepackage{multirow}
    \usepackage{float}
    \usepackage[utf8]{inputenc}

    %++++++++++++++++++++++++++++++++++++++++++++++++++++++++++++++++++++++++++++++
    % Espaçamento padrão de todo o documento
    %++++++++++++++++++++++++++++++++++++++++++++++++++++++++++++++++++++++++++++++
    \onehalfspacing

    %singlespacing Para um espaçamento simples
    %onehalfspacing Para um espaçamento de 1,5
    %doublespacing Para um espaçamento duplo

    %++++++++++++++++++++++++++++++++++++++++++++++++++++++++++++++++++++++++++++++
    % Identificacoes (se o trabalho for em inglês, insira os dados em inglês)
    % Para entradas abreviadas de Professora (Profa.) em português escreva: Prof$^\textnormal{a}$.
    %++++++++++++++++++++++++++++++++++++++++++++++++++++++++++++++++++++++++++++++
    \course{Engenharia Aeronáutica e Mecânica} % Programa de PG ou Curso de Graduação
    \area{Engenharia Aeronáutica} % Área de concentração na PG (Não utilizado no caso de TG)

    % Autor do trabalho: Nome Sobrenome
    \authorgender{masc}                     %sexo: masc ou fem
    \author{João Paulo Monteiro Cruvinel da}{Costa}
    \itaauthoraddress{Av. Quinze de Novembro, 850, Apartamento 126}{14.801-030}{Araraquara--SP}

    % Titulo da Tese/Dissertação
    \title{Metodologia para Cálculo de Cargas em Asas Flexíveis para Projeto Conceitual}

    % Orientador
    \advisorgender{masc}                    % masc ou fem
    \advisor{Prof.~Dr.}{Roberto Gil Annes da Silva}{ITA}

    % Coorientador (Caso não haja coorientador, colocar ambas as variáveis \coadvisorgender e \coadvisor comentadas, com um % na frente)
    %\coadvisorgender{fem}									% masc ou fem
    %\coadvisor{Prof$^\textnormal{a}$.~Dr$^\textnormal{a}$.}{Doralice Serra}{OVNI}

    % Pró-reitor da Pós-graduação
    \bossgender{masc}												% masc ou fem
    \boss{Prof.~Dr.}{Pedro Teixeira Lacava}

    %Coordenador do curso no caso de TG
    %\bosscoursegender{masc}									% masc ou fem
    %\bosscourse{Prof.~Dr.}{John Walker}

    % Palavras-Chaves informadas pela Biblioteca -> utilizada na CIP
    \kwcip{PALAVRA-CHAVE}
    \kwcip{PALAVRA-CHAVE}
    \kwcip{PALAVRA-CHAVE}

    % membros da banca examinadora

    \examiner{Prof. Dr.}{PRESIDENTE}{Presidente}{ITA}
    \examiner{Prof. Dr.}{MEMBRO}{}{INSTITUIÇÂO}
    \examiner{Prof. Dr.}{MEMBRO}{}{INSTITUIÇÂO}
    \examiner{Prof. Dr.}{MEMBRO}{}{INSTITUIÇÂO}
    \examiner{Prof. Dr.}{MEMBRO}{}{INSTITUIÇÂO}

    % Data da defesa (mês em maiúsculo, se trabalho em inglês, e minúsculo se trabalho em português)
    \date{DIA}{MÊS}{ANO}

    % Número CDU - (somente para TG)
    \cdu{621.38}

    % Glossario
    \makeglossary
    \frontmatter

    \begin{document}
    % Folha de Rosto e Capa para o caso do TG
    \maketitle

    % Dedicatoria: Nao esqueca essa secao  ... :-)
    \begin{itadedication}
        DEDICATÓRIA
    \end{itadedication}

    % Agradecimentos
    \begin{itathanks}
    \input{Cap0/agradecimentos}
    \end{itathanks}

    % Epígrafe
    \thispagestyle{empty}
    \ifhyperref\pdfbookmark[0]{\nameepigraphe}{epigrafe}\fi
    \begin{flushright}
    \begin{spacing}{1}
    \mbox{}\vfill
    {\sffamily\itshape
    ``TEXTO EPÍGRAFE\\}
    --- \textsc{FONTE}
    \end{spacing}
    \end{flushright}

    % Resumo
    \begin{abstract}
    \noindent
    \input{Cap0/resumo}
    \end{abstract}

    % Abstract
    \begin{englishabstract}
    \noindent
    \input{Cap0/abstract}
    \end{englishabstract}

    % Lista de figuras
    \listoffigures %opcional

    % Lista de tabelas
    \listoftables %opcional

    % Lista de abreviaturas
    \listofabbreviations
    \begin{longtable}{ll}

    AOA & angle of attack \\

\end{longtable} %opcional

    % Lista de simbolos
    \listofsymbols
    \begin{longtable}{ll}

    $C_l$ & Coeficiente de sustentação \\

\end{longtable} %opcional

    % Sumario
    \tableofcontents

    \mainmatter
    % Os capitulos comecam aqui

    \chapter{CAPÍTULO 001}
    \section{Capítulo 1}
Texto do Capítulo 1

    \chapter{CAPÍTULO 002}
    \section{Capítulo 2}
Texto do Capítulo 2.

    \chapter{CAPÍTULO 003}
    \section{Capítulo 3}
Texto do Capítulo 3.

    \chapter{CAPÍTULO 004}
    \section{Capítulo 4}
Texto do Capítulo 4.

    % REFERENCIAS BIBLIOGRAFICAS
    \renewcommand\bibname{\itareferencesnamebabel} %renomear título do capítulo referências
    \bibliography{Referencias/referencias}

    % Apendices
    \appendix
    \chapter{APÊNDICE 1} %opcional
    \section{Primeiro Apêndice}
Texto do primeiro apêndice.

    % Anexos
    \annex
    \chapter{ANEXO 1} %opcional
    \section{Primeiro Anexo}
Texto do primeiro anexo.

    % Glossario
    %\itaglossary
    %\printglossary

    % Folha de Registro do Documento
    % Valores dos campos do formulario
    \FRDitadata{DATA DE RESGISTRO}
    \FRDitadocnro{NÚMERO DE REGISTRO} %(o número de registro você solicita a biblioteca)
    \FRDitaorgaointerno{Instituto Tecnológico de Aeronáutica -- ITA}
    %Exemplo no caso de pós-graduação: Instituto Tecnol{\'o}gico de Aeron{\'a}utica -- ITA
    \FRDitapalavrasautor{PALAVRA-CHAVE; PALAVRA-CHAVE; PALAVRA-CHAVE}
    \FRDitapalavrasresult{PALAVRA-CHAVE; PALAVRA-CHAVE; PALAVRA-CHAVE}
    %Exemplo no caso de graduação (TG):
    %\FRDitapalavraapresentacao{Trabalho de Graduação, ITA, São José dos Campos, 2015. \NumPenultimaPagina\ páginas.}
    %Exemplo no caso de pós-graduação (msc, dsc):
    \FRDitapalavraapresentacao{ITA, São José dos Campos. Curso de Mestrado Profissionalizante. Programa de Pós-Graduação em Engenharia Aeronáutica e Mecânica. Área de Engenharia Aeronáutica. Orientador: Prof.~Dr. Roberto Gil Annes da Silva. Defesa em DATA-DA-DEFESA. Publicada em DATA-DA-PUBLICAÇÃO.}
    \FRDitaresumo{\input{Cap0/resumo}}
    %  Primeiro Parametro: Nacional ou Internacional -- N/I
    %  Segundo parametro: Ostensivo, Reservado, Confidencial ou Secreto -- O/R/C/S
    \FRDitaOpcoes{N}{O}
    % Cria o formulario
    \itaFRD

    \end{document}
    % Fim do Documento. O massacre acabou!!! :-)
